\documentclass[12pt]{article}
\title{Introduction into Data Science - Project Plan}
\author{Maximilian Tornow}
\usepackage{amsmath}
\DeclareMathOperator*{\btie}{\bowtie}
\usepackage{graphicx}
\graphicspath{ {images/} }
\usepackage{float}
%\usepackage{apacite}
\usepackage{breakcites}
\bibliographystyle{apalike}

%\usepackage[margin=1cm]{geometry}


\begin{document}
	\maketitle
	\begin{enumerate}
		\item What is the general topic?\\
		Dengue fever is a huge problem in some parts of the US, to adapt to the variying amounts of dengue fever patients provides a big challenge to state authorities. \\
		\item What data will be used?\\
		Weather and environmental data of two US cities, on a monthly basis. Combined with the amount of dengue cases in the regarding city.\\
		\item How will data be processed?\\
		This still needs to be decided in particular. But probably a estimation of most NaN values as means, or removal of the NaN datapoints.\\
		\item What questions will be asked and attempted to answer?\\
		How many dengue fever patients will there be in regards to environmental data ?\\
		Which environmental datafeatures are the most important to forecast the projection ?\\
		\item What analysis techniques or algorithms are planned to be used?\\
		The data will be analyized using mainly supervised learning algorithms (LDA, QDA, SVC) as well as PCA to reduce the dimension of the data set. 
	\end{enumerate}





\end{document}